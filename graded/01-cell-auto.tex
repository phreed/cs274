Spring 2012

01/19/12

CS 274: Modeling & Simulation

G. Biswas

Assignment 1(a)

Due: Thursday, February 2, 2012

Instructions: Submit to the Digital Drop Box on Oak by noon on the date due.
Problem 1: Consider one of the available John Conway’s Game of Life
45 points implementations (either a Java applet or a freeware/shareware version).
Discover patterns that lead to each of the four different classes of behaviors.
Include screen captures of the evolution of the different patterns in time,
and generate an explanation of why the different behaviors emerge and why
you classified it as such.

Class 1 – Steady state
Description: Evolution to stable, homogeneous state, i.e., dynamics die out.

Class 2 – Limit Cycles
Description: Evolution leads to stable or oscillating structures, i.e., dynam-
ics become periodic; effects (changes) only propagate a finite distance
(In other words, local changes to the initial pattern tend to remain local).

Class 3 – Chaotic
Description: Effects (changes) propagate at a finite speed forever. Any sta-
ble structures produced are eventually destroyed by their surroundings.
Future states depend on increasing number of past states.
Disordered initial states may lead to chaotic (but not random) behavior.

Class 4 – Structured Chaos
Description: Evolution leads to complex localized structures (may be class
2 stable or oscillating structures), but the number of steps required to
reach this configuration may be very large even if the initial structure is
relatively simple (e.g., R-pentomino): effects propagate indefinitely in
space.

Details:
Capture screen shots that show evolution of behavior
* Initially show how pattern evolves in 1 step, then 5 steps, 10 steps, 20
steps to see if you get a feel for the emerging structure. Use additional
screenshots to support your discussion of the evolving behavior, whether
periodic or non periodic. Summarize in words the dynamic evolution you
observe, and see if you can justify it in terms of a real-world system (use
your intuition). Use the Game of Life rules to support your answer.

* Determine how many steps to run before you reach a steady behavior or
a periodic behavior (classes 1, 2, and 4 and the rate of evolution in class 3)

* Any general conclusions you can draw from your analyses.

Problem 2: Explain the notion of self-organizing behavior in cellular automata.
5 points

* What are attractors, and what is their role in self-organizing behaviors?

From Wolfram

To proceed in analyzing universality in cellular automata, one must first give more
quantitative definitions of the classes identified above. 
One approach to such definitions is to consider the degree of predictability of
the outcome of cellular automaton evolution,
given knowledge of the initial state. 

For class I cellular automata complete prediction is trivial: 
regardless of the initial state, the system always evolves to 
a unique homogeneous state. 

Class 2 cellular automata have
the feature that the effects of particular site
values propagate only a finite distance, 
that is, only to a finite number of neighboring sites. 
Thus a change in the value of a single
initial site affects only a finite region of sites
around it, even after an infinite number of time steps. 
This behavior, illustrated in Fig. 9, 
implies that prediction of a particular final
site value requires knowledge of only a finite
set of initial site values. 

In contrast, changes of initial site values in 
class 3 cellular automata, again as illustrated in Fig. 9, 
almost always propagate at a finite speed forever
and therefore affect more and more distant
sites as time goes on. 
The value of a particular site after many time steps thus
depends on an ever-increasing number of initial site values. 
If the initial state is disordered, this dependence 
may lead to an apparently chaotic succession of values for a
particular site. 
In class 3 cellular automata, therefore, 
prediction of the value of a site at
infinite time would require knowledge of an
infinite number of initial site values. 

Class 4 cellular automata are distinguished by an
even greater degree of unpredictability, as
discussed below.

Class 2 cellular automata may be considered as "filters" that select particular
features of the initial state. 
For example, a class 2 cellular automata may be 
constructed in which initial sequences III survive, 
but sites not in such sequences eventually attain value O. 
Such cellular automata are of practical importance 
for digital image processing:
they may be used to select and enhance
particular patterns of pixels. 
After a sufficiently long time any class 2 cellular 
automaton evolves to a state consisting of blocks
containing nonzero sites separated by regions of zero sites. 
The blocks have a simple form, typically consisting 
of repetitions of particular site values or sequences of site
values (such as 101010 .. .). 
The blocks either do not change with time 
(yielding vertical stripes in the patterns of Fig. 8) 
or cycle between a few states 
(yielding " railroad track" patterns).
While class 2 cellular automata evolve to
give persistent structures with small periods,
class 3 cellular automata exhibit chaotic
aperiodic behavior, as shown in Fig. 8.
Although chaotic, the pattern s generated by
class 3 cellular automata are not completely random. 
In fact, as mentioned for the example of Eq. I, 
they may exhibit important self-organizing behavior. 
In addition and again in contrast to class 2 cellular automata, 
the statistical properties of the states generated
by many time steps of class 3 cellular
automaton evolution are the same for almost
all possible initial states.
The large-time behavior of a class 3 cellular automaton is
therefore determined by these common statistical properties.
The configurations of an infinite cellular
automaton consist of an infinite sequence of
site values. 
These site values could be considered as digits in a real number, 
so that each complete configuration would correspond to a single real number. 
The topology of the real numbers is, however, not
exactly the same as the natural one for the configurations 
(the binary numbers 0.111111 . . . and 1.00000 . .. are identical,
but the corresponding configurations are not). 
Instead, the configurations of an infinite
cellular automaton form a Cantor set. 
Figure 10 illustrates two constructions for a Cantor set.
In construction (a) of Fig. 10, one starts with 
the set of real numbers in the interval 0 to I. 
First one excludes the middle third of the interval, 
then the middle third of each interval remaining, and so on. 
In the limit the set consists of an infinite number of disconnected points. 
If positions in the interval are
represented by ternimals (base 3 fractions,
analogous to base 10 decimals), then the
construction is seen to retain only points
whose positions are represented by ternimals
containing no I's (the point 0.2202022 is
therefore included; 0.2201022 is excluded).
An important featur e of the limiting set is its
self-similarity, or fractal form: a piece of the
set, when magnified, is indistinguishable
from the whole. This self-similarity is math
ematically analogous to that found for the
limiting two-dimensional pattern of Fig. 3.
In construction (b) of Fig. 10, the Cantor
set is formed from the "leaves" of an infinite
binary tree. Each point in the set is reached
by a unique path from the "root" (top as
drawn) of the tree. This path is specified by
an infinite sequence of binary digits, in which
successive digits determine whether the left-
or right-hand branch is taken at each suc-
cessive level in the tree. Each point in the
Cantor set corresponds uniquely to one
infinite sequence of digits and thus to one
configuration of an infinite cellular automa-
ton. Evolution of the cellular automaton then
corresponds to iterated mappings of the
Cantor set to itself. (The locality of cellular
automaton rules implies that the mappings
are continuous.) This interpretation of cellu-
lar automata leads to analogies with the
theory of iterated mappings of intervals of
the real line. (See Mitchell J. Feigenbaum,
"Universal Behavior in Nonlinear Systems,"
Los Alamos Science , Vol. I, No. 1(1980):
4-27.)
Cantor sets are parameterized by their
"dimensions." A convenient definition of
dimension, based on construction (a) of Fig.
10, is as follows. Divide the interval from 0
to 1 into k n bins, each of width k - n. Then let
N(n) be the number of these bins that
contain points in the set. For large n this
number behaves according to
2
o
o
2
2
o 2 o 2 o 2 o 2
02 02 02 02 02 02 02 02
(a)
(b)
N(n) ~
0
n ,
(2)
and d is defined as the "set dimension" of the
Cantor set. If a set contained all points in the
interval 0 to 1, then with this definition its
dimension would simply be 1. Similarly, any
finite number of segments of the real line
would form a set with dimension I. How-
ever, the Cantor set of construction (a),
which contains an infinite number of discon-
nected pieces, has a dimension according to
Eq. 2 of log32 ~ 0.63.
An alternative definition of dimension,
agreeing with the previous one for present

